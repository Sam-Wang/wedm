%============================= appendix.tex ====================================
\Chapter{Time Averaging Modelling Technique}
\label{app:modelling}
	The large signal representation of voltage source in figure \ref{fig:working-4} as derived in equation \eqref{eq:mod14}
	\begin{align*}
		\begin{split}
			\dot{x} &= [dA_1+(1-d)A_2]x + [dB_1 + (1-d)B_2]V_d\\
			V_o &= [dC_1+(1-d)C_2]x
		\end{split}
	\end{align*}

	Introducing small perturbations in $x$, $V_o$ and $d$ as
	\begin{equation}
		\begin{split}
			x &= X + \hat{x}\\
			v_o &= V_o + \hat{v_o}\\
			d &= D + \hat{d}
		\end{split}
		\label{eq:mod15}
	\end{equation} 
	Since the transfer function is to be determined between $\hat{v_o}$ and $\hat{d}$, perturbations in $v_d$ are assumed to be zero for simplicity, therefore
	\begin{equation}
		v_d = V_d
		\label{eq:mod16}
	\end{equation}
	Using the fact that at steady state $\dot{X} = 0$ and the equations \eqref{eq:mod15} in equation \eqref{eq:mod14}
	\begin{equation}
		\dot{\hat{x}} = AX + BV_d + A\hat{x} + [(A_1-A_2)X+(B_1-B_2)V_d]\hat{d} + \text{terms with products of $\hat{x}$ and $\hat{d}$ (neglected)} 
		\label{eq:mod17}
	\end{equation}
	where
	\begin{align}
		A &= A_1D+A_2(1-D)\\
		B &= B_1D+B_2(1-D)
		\label{eq:mod18}
	\end{align}
	At steady state, the perturbations in equation \eqref{eq:mod17} are zero, therefore
	\begin{equation}
		AX+BV_d=0
		\label{eq:mod19}
	\end{equation}
	Hence, when perturbations are introduced in a converter operating under steady state
	\begin{equation}
		\dot{\hat{x}} = A\hat{x} + [(A_1-A_2)X+(B_1-B_2)V_d]\hat{d}
		\label{eq:mod20}
	\end{equation}

	Similarly, for output voltage, using equations \eqref{eq:mod15} in equation \eqref{eq:mod14}
	\begin{equation}
		V_o + \hat{v}_o = CX + C\hat{x} + [(C_1-C_2)X]\hat{d}
		\label{eq:mod21}
	\end{equation}
	where
	\begin{equation}
		C = C_1D+C_2(1-D)
		\label{eq:mod22}
	\end{equation}
	At steady state
	\begin{equation}
		V_o = CX
		\label{eq:mod23}
	\end{equation}
	Hence, when perturbations are introduced in a converter operating under steady state
		\begin{equation}
		\hat{v}_o = C\hat{x}+[(C_1-C_2)X]\hat{d}
		\label{eq:mod24}
	\end{equation}
	Taking Laplace transform of equation \eqref{eq:mod20}
	\begin{align}
		s\hat{x}(s) &= Ax(s)+[(A_1-A_2)X+(B_1-B_2)V_d]\hat{d}(s)\\
		\therefore \hat{x}(s) &= [sI-A]^{-1}[(A_1-A_2)X+(B_1-B_2)V_d]\hat{d}(s)
		\label{eq:mod25}
	\end{align}
	where $I$ is a unity matrix
	Taking Laplace transform of equation \eqref{eq:mod24}
	\begin{equation}
		\hat{v}_o(s) = C\hat{x}(s)+[(C_1-C_2)X]\hat{d}(s)
		\label{eq:mod26}
	\end{equation}
	Substituting in $\hat{x}(s)$ from equation \eqref{eq:mod25}
	\begin{equation}
		\hat{v}_o(s) = \{C[sI-A]^{-1}[(A_1-A_2)X+(B_1-B_2)V_d]+(C_1-C_2)X\}\hat{d}(s)
		\label{eq:mod27b}
	\end{equation}
	\begin{equation}
		\therefore \dfrac{\hat{v}_o(s)}{\hat{d}(s)} = C[sI-A]^{-1}[(A_1-A_2)X+(B_1-B_2)V_d]+(C_1-C_2)X
		\label{eq:mod27}
	\end{equation}

	This is the small signal transfer function of the output voltage of two quadrant converter with respect to the duty ratio of the switch $Q_2$
%===============================================================================